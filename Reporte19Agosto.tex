\documentclass[12pt,letterpaper]{article}
\usepackage[utf8]{inputenc}
\usepackage[T1]{fontenc}
\usepackage{amsmath}
\usepackage{amssymb}
\usepackage{graphicx}
\usepackage[spanish]{babel}
\usepackage[margin=1 in]{geometry}
\usepackage{multicol}
\title{Reporte: Tesis I}
\author{Lic. Arnoldo Del Toro Peña}
\begin{document}
	\maketitle
	
	\begin{abstract}
		En un problema de hormigón se tiene un enfrentamiento en distintos puntos de vista del problema, por un lado tenemos la adquisición de hormigón, el transporte de este material o la asignación de los vehículos en las distintas rutas. El problema se centra en encontrar rutas eficientes para una flota de vehículos alternando entre centros de producción y lugares de construcción de hormigón, restricciones de enrutamiento y programación estricta. Anteriormente se ha trabajado un Modelo de Programación Entero Mixto (MIP) con modelo de programación de restricciones.
	\end{abstract}

	\begin{multicols}{2}
		
		\section{Introducci\'on}
		Cada uno de los clientes tienen un determinado lugar para recibir el pedido, además existe una ventana de tiempo $[a_i,b_i]$ que denota el intervalo de tiempo durante cada entrega puede realizarse. El Problema de repartición de concreto (CDP), es una combinación de los problemas: planificación y enrutamiento. El concreto es producido en sitios de producción de hormigón localizados a una distancia de los clientes.
		\section{Metodolog\'ia}
		En 
		\section{Marco Te\'orico}
		
		\section{Conclusiones}
		
	\end{multicols}

	\section{Bibliograf\'ia}
	
	
\end{document}