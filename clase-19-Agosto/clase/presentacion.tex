\documentclass{beamer}
\usepackage[spanish]{babel}
\usepackage{tikz} 
\usepackage{tikz-network}
\usepackage[utf8]{inputenc}
\usepackage{algorithm,algorithmic}
\usetheme{CambridgeUS}
\usepackage{url}
\usepackage{hyperref}
\usepackage{natbib} 
\title{Presentación Tesis I}
\author{Lic. Arnoldo Del Toro Peña }
\institute[UANL]{Universidad Autónoma de Nuevo León}
\titlegraphic{ \includegraphics[width=2cm]{R.jpg} \hfill \includegraphics[width=4cm]{fime.png} }

\newtheorem{definicion}{Definición}
\newcommand{\ql}{Q-Learning}

\setbeamertemplate{bibliography entry title}{}
\setbeamertemplate{bibliography entry location}{}
\setbeamertemplate{bibliography entry note}{}

%\usepackage{pgfpages}
%\setbeameroption{show notes on second screen=right} estas lineas se activan cuando la presentación será en un monitor externo.

\begin{document}
	
	\begin{frame}
		\titlepage
	\end{frame}
	
	\begin{frame}{Tabla de contenidos}
		\tableofcontents
	\end{frame}
	
	\section{Introducción}
	
	\begin{frame}{Introducción}
		En un problema de hormigón se tiene un enfrentamiento en distintos puntos de vista del problema.
		
		\begin{enumerate}
			\item Vehículos.
			\item Rutas.
			\item Ventanas de tiempo.
			\item Clientes.
		\end{enumerate}
	
	\end{frame} 
	
	\begin{frame}{Vehículos}
		Características de los vehículos:
		\begin{enumerate}
			\item Carga.
			\item Tiempo de llegada y salida.
		\end{enumerate}
	\end{frame}

	\begin{frame}{Rutas}
		Las rutas es una parte importante a determinar, ya que se tiene una cantidad de clientes a satisfacer y varios centros de abastecimiento en los cuales el vehículo reabastecer su carga.
	\end{frame}
	
	\begin{frame}{Ventanas de tiempo}
		La mayor problemática del problema.
		Restricciones a tomar:
		\begin{enumerate}
			\item Hora de salida del centro de abastecimiento.
			\item Hora de llegada con cliente.
			\item Tiempo de espera para recibir el último envío.
			\item Número de envíos.
		\end{enumerate}
	\end{frame}
	
	\begin{frame}{\ql}
		Actualmente se tienen resultado tomando acciones aleatorias.
		\newline
		Se esta trabajando en la implementación de acciones no aleatorias.
		\newline
		En un futuro cercano implementaremos acciones en base a heurísticas.
	\end{frame}
		
	
	


	

\end{document}
